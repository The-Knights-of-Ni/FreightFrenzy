%% Generated by Sphinx.
\def\sphinxdocclass{report}
\documentclass[letterpaper,10pt,english]{sphinxmanual}
\ifdefined\pdfpxdimen
   \let\sphinxpxdimen\pdfpxdimen\else\newdimen\sphinxpxdimen
\fi \sphinxpxdimen=.75bp\relax
\ifdefined\pdfimageresolution
    \pdfimageresolution= \numexpr \dimexpr1in\relax/\sphinxpxdimen\relax
\fi
%% let collapsible pdf bookmarks panel have high depth per default
\PassOptionsToPackage{bookmarksdepth=5}{hyperref}

\PassOptionsToPackage{warn}{textcomp}
\usepackage[utf8]{inputenc}
\ifdefined\DeclareUnicodeCharacter
% support both utf8 and utf8x syntaxes
  \ifdefined\DeclareUnicodeCharacterAsOptional
    \def\sphinxDUC#1{\DeclareUnicodeCharacter{"#1}}
  \else
    \let\sphinxDUC\DeclareUnicodeCharacter
  \fi
  \sphinxDUC{00A0}{\nobreakspace}
  \sphinxDUC{2500}{\sphinxunichar{2500}}
  \sphinxDUC{2502}{\sphinxunichar{2502}}
  \sphinxDUC{2514}{\sphinxunichar{2514}}
  \sphinxDUC{251C}{\sphinxunichar{251C}}
  \sphinxDUC{2572}{\textbackslash}
\fi
\usepackage{cmap}
\usepackage[T1]{fontenc}
\usepackage{amsmath,amssymb,amstext}
\usepackage{babel}



\usepackage{tgtermes}
\usepackage{tgheros}
\renewcommand{\ttdefault}{txtt}



\usepackage[Bjarne]{fncychap}
\usepackage{sphinx}

\fvset{fontsize=auto}
\usepackage{geometry}


% Include hyperref last.
\usepackage{hyperref}
% Fix anchor placement for figures with captions.
\usepackage{hypcap}% it must be loaded after hyperref.
% Set up styles of URL: it should be placed after hyperref.
\urlstyle{same}


\usepackage{sphinxmessages}




\title{Freight Frenzy}
\date{Dec 30, 2021}
\release{0.0.0\sphinxhyphen{}alpha}
\author{The Knights of Ni}
\newcommand{\sphinxlogo}{\vbox{}}
\renewcommand{\releasename}{Release}
\makeindex
\begin{document}

\pagestyle{empty}
\sphinxmaketitle
\pagestyle{plain}
\sphinxtableofcontents
\pagestyle{normal}
\phantomsection\label{\detokenize{index::doc}}


\sphinxAtStartPar
If you are looking to see what a game pad key does or how to configure the robot, look at the quickstart guide.
If there are errors with the code, look at the common errors page. If you want detailed instruction on how to use the
robot take a look at {\hyperref[\detokenize{configuration::doc}]{\sphinxcrossref{\DUrole{doc}{Configuration}}}} and {\hyperref[\detokenize{game_pad::doc}]{\sphinxcrossref{\DUrole{doc}{Game Pad}}}}.


\chapter{Contents}
\label{\detokenize{index:contents}}

\section{Quickstart}
\label{\detokenize{quickstart:quickstart}}\label{\detokenize{quickstart::doc}}
\sphinxAtStartPar
This guide will help you quickstart the robot.


\subsection{Game Pad Mapping}
\label{\detokenize{quickstart:game-pad-mapping}}
\sphinxAtStartPar
To start the robot press the Start key, and then the A key.

\sphinxAtStartPar
Joystick 1 \sphinxhyphen{} Up and down are forwards and backwards. Left and right strafe.

\sphinxAtStartPar
X, Y \sphinxhyphen{} Duck wheel with direction

\sphinxAtStartPar
A, B \sphinxhyphen{} Intake motor with direction


\subsection{Robot Configuration}
\label{\detokenize{quickstart:robot-configuration}}
\sphinxAtStartPar
Most of the configuration of the robot in done in \sphinxcode{\sphinxupquote{/teamcode/org/ftc/firstinspires/teamcode/Config/MainConfig.java}}


\section{Configuration}
\label{\detokenize{configuration:configuration}}\label{\detokenize{configuration::doc}}
\sphinxAtStartPar
This discusses the configuration files (in \sphinxcode{\sphinxupquote{/teamcode/org/ftc/firstinspires/teamcode/Config/}}) that you might have to edit, the others do not have to be edited.


\subsection{Main Configuration}
\label{\detokenize{configuration:main-configuration}}
\sphinxAtStartPar
\sphinxcode{\sphinxupquote{name}} \sphinxhyphen{} The robot’s name

\sphinxAtStartPar
\sphinxcode{\sphinxupquote{version}} \sphinxhyphen{} The robot’s version

\sphinxAtStartPar
\sphinxcode{\sphinxupquote{allianceColor}} \sphinxhyphen{} The alliance color, it’s either AllianceColor.BLUE or AllianceColor.RED.

\sphinxAtStartPar
\sphinxcode{\sphinxupquote{debug}} \sphinxhyphen{} Either true or false, at the current moment it does not do anything

\sphinxAtStartPar
\sphinxcode{\sphinxupquote{debugTarget}} \sphinxhyphen{} “none” means no target enter a file path to debug a file enter a folder path to debug a folder, “*” means
debug all

\sphinxAtStartPar
\sphinxcode{\sphinxupquote{logLevel}} \sphinxhyphen{} 0 is quiet, the maximum is 5.

\sphinxAtStartPar
\sphinxcode{\sphinxupquote{initSubsystems}} \sphinxhyphen{} Whether to initialize the subsystems, it might break stuff if it is false.

\sphinxAtStartPar
\sphinxcode{\sphinxupquote{initSubsystemControl}} \sphinxhyphen{} Whether to initialize control (It might crash).

\sphinxAtStartPar
\sphinxcode{\sphinxupquote{initSubsystemDrive}} \sphinxhyphen{} Whether to initialize drive.

\sphinxAtStartPar
\sphinxcode{\sphinxupquote{initSubsystemVision}} \sphinxhyphen{} Whether to initialize vision.

\sphinxAtStartPar
\sphinxcode{\sphinxupquote{initMechanical}} \sphinxhyphen{} Set to false if you don’t want to init Mechanical stuff

\sphinxAtStartPar
\sphinxcode{\sphinxupquote{initGetGamePadInputs}} \sphinxhyphen{} Set to false if you are not using the game pad.

\sphinxAtStartPar
\sphinxcode{\sphinxupquote{initHardwareMap}} \sphinxhyphen{} Set to false if you do not want to init the hardware map.


\subsection{Vision Configuration}
\label{\detokenize{configuration:vision-configuration}}
\sphinxAtStartPar
\sphinxcode{\sphinxupquote{CAMERA\_WIDTH}} \sphinxhyphen{} the width of the wanted camera resolution

\sphinxAtStartPar
\sphinxcode{\sphinxupquote{CAMERA\_HEIGHT}} \sphinxhyphen{} the height of the wanted camera resolution

\sphinxAtStartPar
\sphinxcode{\sphinxupquote{HORIZON}} \sphinxhyphen{} the horizon value to tune

\sphinxAtStartPar
\sphinxcode{\sphinxupquote{WEBCAM\_NAME}} \sphinxhyphen{} the name of the webcam

\sphinxAtStartPar
\sphinxcode{\sphinxupquote{VUFORIA\_KEY}} \sphinxhyphen{} The Vuforia Key

\sphinxAtStartPar
\sphinxcode{\sphinxupquote{finalMarkerLocation}} \sphinxhyphen{} Where the marker is


\section{Game Pad}
\label{\detokenize{game_pad:game-pad}}\label{\detokenize{game_pad::doc}}

\subsection{Game Pad 1}
\label{\detokenize{game_pad:game-pad-1}}
\noindent{\hspace*{\fill}\sphinxincludegraphics[scale=0.3]{{xbox-360-controller-front}.jpg}\hspace*{\fill}}

\sphinxAtStartPar
Left Joystick \sphinxhyphen{} Movement
Right Joystick \sphinxhyphen{} Turning
A \sphinxhyphen{} Intake counter clockwise
B \sphinxhyphen{} Intake clockwise
X \sphinxhyphen{}  Bucket Up
Y \sphinxhyphen{} Bucket Down


\subsection{Gamepad 2}
\label{\detokenize{game_pad:gamepad-2}}
\sphinxAtStartPar
A \sphinxhyphen{} Linear Slide Level 3
B \sphinxhyphen{} Linear Slide Retract
X \sphinxhyphen{} Duck Wheel Toggle Forward
Y \sphinxhyphen{} Duck Wheel Toggle Backward
Left Bumper \sphinxhyphen{} Claw Retract
Right Bumper \sphinxhyphen{} Claw Extend


\section{Common Errors}
\label{\detokenize{common_errors:common-errors}}\label{\detokenize{common_errors::doc}}

\subsection{Out of Memory error}
\label{\detokenize{common_errors:out-of-memory-error}}
\sphinxAtStartPar
Try to compile again, if that fails increase java vm size, using \sphinxcode{\sphinxupquote{gradle.properties}}.


\subsection{Java 1.8 Error}
\label{\detokenize{common_errors:java-1-8-error}}
\sphinxAtStartPar
Follow the suggested steps, start by checking you android studio preferences.


\chapter{Indices and tables}
\label{\detokenize{index:indices-and-tables}}\begin{itemize}
\item {} 
\sphinxAtStartPar
\DUrole{xref,std,std-ref}{genindex}

\item {} 
\sphinxAtStartPar
\DUrole{xref,std,std-ref}{modindex}

\item {} 
\sphinxAtStartPar
\DUrole{xref,std,std-ref}{search}

\end{itemize}



\renewcommand{\indexname}{Index}
\printindex
\end{document}